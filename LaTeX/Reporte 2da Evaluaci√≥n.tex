\documentclass[french]{article}
\usepackage[T1]{fontenc}
\usepackage[utf8]{inputenc}
\usepackage{lmodern}
\usepackage[a4paper]{geometry}
\usepackage[spanish]{babel}
\usepackage{graphicx}
\title{Reporte de la segunda evaluación práctica}
\author{David Alonso Garduño Granados}
\begin{document}
	\maketitle
	
Al inicio, me sentía bastante abrumado por la cantidad de información que intentaba manejar el documento y por algunas situaciones personales, decidí dejarlo hasta el último momento.

Parecía complicado entender como funcionaba esta nueva noción del seno, pero eran operaciones hasta llegar a una aproximación, al inicio intenté condicionar un while para lograr las iteraciones necesarias sobre las operaciones que marcaba el documento, luego intenté meter un rango que me ayudara, al final, noté que me estaba complicando de más, pues con una lista definida y un buen indice el cual usar sobre las operaciones que lo necesitaban se solucionaba.

Otro de los problemas con los que me encontré fue la construcción de la lista, pues construir una igual al del documento era complicado, así que aproveché la ambiguedad del enunciado para construir una con solo los valores de las operaciones que hacía cada iteración.

Los enunciados que tenía el archivo de classroom eran bastante claros en sus funciones.
\end{document}
